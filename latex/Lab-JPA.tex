\documentclass[12pt]{scrartcl}

\setlength{\parindent}{0pt}
\setlength{\parskip}{.25cm}

\usepackage{graphicx}

\usepackage{xcolor}

\definecolor{darkred}{rgb}{0.5,0,0}
\definecolor{darkgreen}{rgb}{0,0.5,0}
\usepackage{hyperref}
\hypersetup{
  letterpaper,
  colorlinks,
  linkcolor=red,
  citecolor=darkgreen,
  menucolor=darkred,
  urlcolor=blue,
  pdfpagemode=none,
  pdftitle={Introduction To Git},
  pdfauthor={Christopher M. Bourke},
  pdfcreator={$ $Id: cv-us.tex,v 1.28 2009/01/01 00:00:00 cbourke Exp $ $},
  pdfsubject={PhD Thesis},
  pdfkeywords={}
}

\definecolor{MyDarkBlue}{rgb}{0,0.08,0.45}
\definecolor{MyDarkRed}{rgb}{0.45,0.08,0}
\definecolor{MyDarkGreen}{rgb}{0.08,0.45,0.08}

\definecolor{mintedBackground}{rgb}{0.95,0.95,0.95}
\definecolor{mintedInlineBackground}{rgb}{.90,.90,1}

%\usepackage{newfloat}
\usepackage[newfloat=true]{minted}
\setminted{mathescape,
               linenos,
               autogobble,
               frame=none,
               framesep=2mm,
               framerule=0.4pt,
               %label=foo,
               xleftmargin=2em,
               xrightmargin=0em,
               startinline=true,  %PHP only, allow it to omit the PHP Tags *** with this option, variables using dollar sign in comments are treated as latex math
               numbersep=10pt, %gap between line numbers and start of line
               style=default, %syntax highlighting style, default is "default"
               			    %gallery: http://help.farbox.com/pygments.html
			    	    %list available: pygmentize -L styles
               bgcolor=mintedBackground} %prevents breaking across pages
               
\setmintedinline{bgcolor={mintedBackground}}
\setminted[text]{bgcolor={mintedBackground},linenos=false,autogobble,xleftmargin=1em}
%\setminted[php]{bgcolor=mintedBackgroundPHP} %startinline=True}
\SetupFloatingEnvironment{listing}{name=Code Sample}
\SetupFloatingEnvironment{listing}{listname=List of Code Samples}

\usepackage{tikz}
\usetikzlibrary{calc,shapes.multipart,chains,arrows}

\title{CSCE 156 -- Computer Science II}
\subtitle{Lab 10.5 - Java Persistence API (JPA)}
\author{Dr.\ Chris Bourke}
\date{~}

\begin{document}

\maketitle

\section*{Prior to Lab}

\begin{enumerate}
  \item Review the lecture notes and examples of JPA
  \item Read the following tutorial on JPA: \url{http://docs.oracle.com/javaee/6/tutorial/doc/bnbpz.html}
\end{enumerate}

\section*{Lab Objectives \& Topics}
Following the lab, you should be able to:
\begin{itemize}
  \item Understand the basics of JPA
  \item Use JPA to annotate Java classes to facilitate loading 
    data
  \item Use JPA to persist data to a database
\end{itemize}

\section*{Peer Programming Pair-Up}

To encourage collaboration and a team environment, labs will be
structured in a \emph{pair programming} setup.  At the start of
each lab, you will be randomly paired up with another student 
(conflicts such as absences will be dealt with by the lab instructor).
One of you will be designated the \emph{driver} and the other
the \emph{navigator}.  

The navigator will be responsible for reading the instructions and
telling the driver what to do next.  The driver will be in charge of the
keyboard and workstation.  Both driver and navigator are responsible
for suggesting fixes and solutions together.  Neither the navigator
nor the driver is ``in charge.''  Beyond your immediate pairing, you
are encouraged to help and interact and with other pairs in the lab.

Each week you should alternate: if you were a driver last week, 
be a navigator next, etc.  Resolve any issues (you were both drivers
last week) within your pair.  Ask the lab instructor to resolve issues
only when you cannot come to a consensus.  

Because of the peer programming setup of labs, it is absolutely 
essential that you complete any pre-lab activities and familiarize
yourself with the handouts prior to coming to lab.  Failure to do
so will negatively impact your ability to collaborate and work with 
others which may mean that you will not be able to complete the
lab.  

\section*{Java Persistence API}

TODO

\section*{Activities}

Clone the starter code for this lab from GitHub using the following
url: \url{https://github.com/cbourke/TODO}.

\section*{Configuration}

\begin{itemize}
  \item Be sure that you have the albums database installed on your CSE 
    database
  \item Open the \mintinline{text}{config/META-INF/persistence.xml} file.
  	This is the primary configuration file that will allow you to define
	``persistence units,'' which are essentially database profiles.
  \item We have already defined a persistence unit named \mintinline{text}{"album_database"}.
    Change the \mintinline{text}{url}, \mintinline{text}{username} and
    \mintinline{text}{password} values to reflect your database 
    credentials.
\end{itemize}

\section*{Troubleshooting}

Open the \mintinline{java}{Part1} class and run the code.  There are several
problems that you will need to address before it works properly.

\begin{enumerate}
  \item The first time you run it, the exception will be self-explanatory, 
  	fix the issue and rerun it.
  \item The second time you run it, there will be a different error.  Observe
  	that in the \mintinline{java}{DataLoader} class, the \mintinline{java}{EntityManager}
	is closed before each album's band can be loaded from the data base.  Focus
	on the annotations responsible for the \mintinline{java}{Album} $\rightarrow$ 
	\mintinline{java}{Band} relation and fix the problem.
\end{enumerate}

\section*{Annotating a Class}

Open the \mintinline{java}{Part2} class and run the code.  It should print
album information, but it does not print the songs on each album.  This is
because we have not annotated the \mintinline{java}{AlbumSong} ``join class''
and associated the entity with the \mintinline{java}{List<AlbumSong> songs}
member variable in the \mintinline{java}{Album} class.

\begin{enumerate}
  \item First, remove the \mintinline{java}{@Transient} annotation on the
  \mintinline{java}{songs} variable in the \mintinline{java}{Album} class\footnote{We
  only placed this annotation on the class to get the first demo to work.} and
  replace it with the following annotations:

  \begin{minted}{java}
  @OneToMany(fetch=FetchType.EAGER)
  @JoinColumn(name="albumId", nullable=false)
  @OrderBy("trackNumber ASC")
  \end{minted}
  
  Run the demo again and observe the results.
  \item Now annotate the \mintinline{java}{AlbumSong} class to make it an 
  entity.  
  \item Contrast the way we associated the \mintinline{java}{Album} and 
  \mintinline{java}{Song} classes through the \mintinline{java}{AlbumSong}
  ``join class'' and the way we associated the \mintinline{java}{Band} and
  \mintinline{java}{Musician} classes.  Both are many-to-many
  relationships with join tables, but one has a join class and the other
  does not.  Discuss with your partner the consequences of these design choices.  
\end{enumerate}

\section*{Querying Using JPQL}

You will now get some practice writing JPA code to load data from a database
using JPQL.  Open the \mintinline{java}{Part3} class.  
\begin{enumerate}
  \item Implement the \mintinline{java}{getBands()} and 
    \mintinline{java}{getBandById(int bandId)} methods in the \mintinline{java}{DataLoader}
    class.  Use the other methods in that class as examples.
  \item Run the \mintinline{java}{Part3} class to troubleshoot your implementations.
\end{enumerate}

\section*{Persisting Data}

You will now use JPA to persist (save) data to the database by implementing the 
\mintinline{java}{addAlbum()} method in the \mintinline{java}{DataPersister} class.
To do this you will need to:
  
\begin{enumerate}
  \item Create an \mintinline{java}{EntityManager} object
  \item Get the \mintinline{java}{EntityManager}'s transaction and begin it.
  \item Persist the given \mintinline{java}{Album} object
  \item Commit the transaction
  \item Catch and handle any exceptions (rolling back if necessary) and
  	clean up your resources.
\end{enumerate}

Test your code using the \mintinline{java}{Part4} class.

\section*{Advanced Activity}

Modify the albums web app (labs 9 and 10) to use your JPA classes.  To get this
to work you will need to:
\begin{itemize}
  \item Make sure that all the required JAR files are in the \mintinline{text}{WebContent/WEB-INF/lib} folder (JAR files in subdirectories will not be found).
  \item Place your \mintinline{text}{persistence.xml} file in the 
  \mintinline{text}{WebContent/WEB-INF/classes/META-INF} folder (you may need to
  manually create this.
\end{itemize}
\end{document}
